\documentclass[12pt]{article}
\usepackage{subfiles}
\usepackage{graphicx}
\usepackage[
        pdfencoding=auto,%
        pdfauthor={Scott Pratt},%
        pdfstartview=FitV,%
        colorlinks=true,%
        linkcolor=blue,%
        citecolor=blue, %
        urlcolor=blue,
        breaklinks=true]{hyperref}
%\usepackage[anythingbreaks,hyphenbreaks]{breakurl}
\usepackage[anythingbreaks,hyphenbreaks]{xurl}
%\usepackage{pdfsync}
\usepackage{amssymb}
\usepackage{amsmath}
\usepackage{bm}
\numberwithin{equation}{section} 
\numberwithin{figure}{section} 
\usepackage[small,bf]{caption}
%\usepackage{fontspec}
%\usepackage{textcomp}
%\usepackage{color}
\usepackage{fancyhdr}
\setlength{\headheight}{16pt}
%\usepackage[headheight=110pt]{geometry}
\usepackage{bm}

\usepackage[most]{tcolorbox}
\tcbset{
frame code={}
center title,
left=0pt,
right=0pt,
top=0pt,
bottom=0pt,
colback=gray!25,
colframe=white,
width=\dimexpr\textwidth\relax,
enlarge left by=0mm,
boxsep=5pt,
arc=0pt,outer arc=0pt,
}
\newcounter{examplecounter}
\counterwithin{examplecounter}{section}
\setcounter{examplecounter}{0}
\newcommand{\example}[2]{\begin{tcolorbox}[breakable,enhanced]
\refstepcounter{examplecounter}{
\bf Example \arabic{section}.\arabic{examplecounter}:}~~{\bf #1}\\
{#2}
\end{tcolorbox}
}
%\newcommand{\exampleend}{
%\begin{samepage}
%\nopagebreak\noindent\rule{\textwidth}{1pt}
%\end{samepage}
%}


%\usepackage{silence}
%\WarningFilter{hyperref}{Token not allowed in a PDF String}

\newcommand\eqnumber{\addtocounter{equation}{1}\tag{\theequation}}

\newcommand{\solution}[1]{ }



\usepackage{comment}
\parskip 4pt
\parindent 0pt

%\newcommand{\bm}{\boldmath}
\boldmath

%

\begin{document}

\today\\

\centerline{\bf\Large Fluctuation Theory}
%\centerline{Oleh Savchuk and Scott Pratt}
%\centerline{Michigan State University}

\section{Sum Rules}
After some effort,
\begin{align*}
\int d\Delta y&\langle \delta E_T(\Delta y)\delta E_T(0)\rangle\\
&=\int d\Delta\eta\langle \delta\epsilon(\Delta\eta)\delta\epsilon(0)+\delta\tilde{P}_z(\Delta\eta)\delta\tilde{P}_z(0)\rangle\\
\eqnumber\label{eq:nocosh}
&=-\chi_{EE}-\chi_{pp}.
\end{align*}


\begin{align*}
\int d\Delta y &\langle \delta E_T(\Delta y)\delta E_T(0)\rangle\cosh(\Delta y)\\
&=\int d\Delta\eta\large\{\langle \delta\epsilon(\Delta\eta)\delta\epsilon(0)-\delta\tilde{P}_z(\Delta\eta)\delta\tilde{P}_z(0)\rangle\cosh(\Delta y)]\\
&~~~+\langle \delta\tilde{P}_z(\Delta\eta)\delta\epsilon(0)-\delta\epsilon(\Delta\eta)\delta\tilde{P}_z(0)\rangle\sinh(\Delta y)\large\}\\
\eqnumber\label{eq:wcosh}
&=-\chi_{EE}+\chi_{pp}.
\end{align*}

The susceptibilities can also be identified as:
\begin{align*}\eqnumber
\chi_{EE}=C_V,~~\chi_{pp}=(P+\epsilon)T.
\end{align*}
Also, the last 2 terms in Eq. (\ref{eq:wcosh} are equal
\begin{align*}\eqnumber
\langle\delta\tilde{P}_z(\Delta\eta)\delta\epsilon(0)\rangle=
-\langle\delta\epsilon(\delta\eta)\delta\tilde{P}_z(0)\rangle.
\end{align*}

Equivalently, one can add  Eq.s(\ref{eq:nocosh}) and (\ref{eq:wcosh}) to get
\begin{align*}
\int d\Delta y&\langle \delta E_T(\Delta y/2)\delta E_T(-\Delta y/2)\rangle\cosh^2(\Delta y/2)\\
&=\int d\Delta\eta\large\{   \langle \delta\epsilon(\Delta\eta/2)\delta\epsilon(\Delta\eta/2)\rangle\cosh^2(\Delta\eta/2)\\
&~~~~~~~~~~~~~~~~-\langle\delta\tilde{P}_z(\Delta\eta/2)\delta\tilde{P}_z(-\Delta\eta/2)\rangle\sinh^2(\Delta\eta/2)\\
&~~~~~~~~~~~~~~~~+\langle \delta\tilde{P}_z(\Delta\eta/2)\delta\epsilon(-\Delta\eta/2)-\delta\epsilon(\Delta\eta/2)\delta\tilde{P}_z(-\Delta\eta/2)\rangle\\
&\hspace*{80pt}\sinh(\Delta \eta/2)\cosh(\Delta \eta/2)\large\}\\
\eqnumber\label{eq:alt1}
&=-\chi_{EE}.
\end{align*}
or subtract them to get
\begin{align*}
\int d\Delta y&\langle \delta E_T(\Delta y/2)\delta E_T(-\Delta y/2)\rangle\sinh^2(\Delta y/2)\\
&=\int d\Delta\eta\large\{\langle\delta\tilde{P}_z(\Delta\eta/2)\delta\tilde{P}_z(-\Delta\eta/2)\rangle\cosh^2(\Delta\eta/2)\\
&~~~~~~~~~~~~~~~~-\langle \delta\epsilon(\Delta\eta/2)\delta\epsilon(-\Delta\eta/2)\rangle\sinh^2(\Delta\eta/2)\\
&~~~~~~~~~~~~~~~~-\langle \delta\tilde{P}_z(\Delta\eta/2)\delta\epsilon(-\Delta\eta/2)-\delta\epsilon(\Delta\eta/2)\delta\tilde{P}_z(-\Delta\eta/2)\rangle\\
&\hspace*{80pt}\sinh(\Delta \eta/2)\cosh(\Delta \eta/2)\large\}\\
\eqnumber\label{eq:alt2}
&=-\chi_{pp}.
\end{align*}
\end{document}
